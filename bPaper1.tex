\documentclass[review,authoryear]{elsarticle} 

\usepackage{lineno}
\usepackage{listings}
\usepackage{array} 
\usepackage{graphicx}
\usepackage{upquote}
\usepackage{booktabs}
\usepackage{xcolor}
\definecolor{ligth-gray}{gray}{0.3}
\usepackage{fancyvrb}
%% \fvset{formatcom =\color{ligth-gray},
\fvset{formatcom =\color{blue},
fontsize=\footnotesize,xleftmargin=10mm}
\RecustomVerbatimEnvironment{verbatim}{Verbatim}{}

\usepackage{caption}
\captionsetup[table]{font=footnotesize,
skip=10pt,labelfont=bf,format=hang,
justification = justified,
margin = 0pt}
\usepackage{amsmath}
\usepackage{tabu}
%% \usepackage{doi}
\usepackage{url}
\usepackage{natbib}
\usepackage{filecontents}
\usepackage{hyperref}
\hypersetup{
  colorlinks,
  citecolor=Violet,
  linkcolor=Red,
  urlcolor=Blue}


\begin{document}
\begin{frontmatter}
%% \title{\textbf{R-package 'biodRy': multilevel modeling of dendro-climatical interactions}}
\title{\textbf{biodRy: an R package to develop multilevel modeling of
dendro-climatical interactions}}

\author[aut1]{W. Lara\corref{cor1}\fnref{fn1}}
%% \author[aut1,aut2]{W. Lara\corref{cor1}\fnref{fn1}}
\author[aut1]{F. Bravo}
%% \author[focal,aut2]{C. A.  Sierra }
\cortext[cor1]{Corresponding author}

\fntext[fn1]{\emph{E-mail address}:
  \href{mailto:wilson.lara@alumnos.uva.es}{wilson.lara@alumnos.uva.es}
  (W.Lara)}

\address[aut1]{Sustainable Forest Management Research
  Institute,UVA-INIA, Avenida Madrid, s/n, 34071, Palencia, Spain}

%% \address[focal]{Department of Biogeochemical Processes, Max Planck
%%   Institute for Biogeochemistry, Hans-Kn\"oll-Stra\ss e 10, 07745,
%%   Jena, Germany}

%% \address[aut2]{Research Center on Ecosystems and Global Change,
%%   Carbono \& Bosques $($C\&B$)$, Calle 51A, N$^o$ 72-23, Int: 601,
%%   050034, Medell{\'i}n, Colombia}

\begin{abstract}
\end{abstract}
\begin{keyword}
\end{keyword}
\end{frontmatter}

\linenumbers
\section{Introduction}\label{sec:intro}


\section{Package requirements and installation}
Performance of {\tt biodRy} has been successfully tested in the binary
distributions of the base system of linux --Debian, Redhat, Suse, and
Ubuntu-- and the contributed packages for Windows and MacIntosh (see
the checks of the package on CRAN). Up to now the package only depends
on two other R packages in CRAN: \verb@nlme@ \citep{Pinheiro2000}, and
\verb@ecodist@\citep{Goslee2007}. Package \verb@nlme@ is used to
extract residuals of the modeled growth with linear mixed effects
models (detrending), and the package \verb@ecodist@ is used to compute
Mantel correlograms between models of tree growth and aridity.

The package can be installed from CRAN within an R session with the
command: \verb@install.packages('biodRy')@. Once the package has been
installed, it can be loaded in R environment with any of the commands:
{\tt require('biodRy')} or {\tt library('biodRy')}. Both
dependences: {\tt nlme} and {\tt ecodist} are automatically loaded in R
environment after the package has been required. We recommend to run
the examples in the package documentation to obtain a better
understanding of the descriptions provided in this manuscript.

\subsection{Inputs}\label{sec:inputs}

{\tt biodRy} processes multilevel data frames (MDFs) containing
serial records in initial columns, followed by recorded times (i.e.,
moths, years, relative times, etc.), and ended with factor-column
levels, with factors being ordered from lower levels (usually a
core-sample replicate, or an annual set of monthly meteorological
records) to higher levels in sampling hierarchy (i.e., plots, sites,
or other spatial units).

Two exaples of MDFs are found in datasets: {\tt Prings05} and {\tt
  PTclim05} provided with the package. Former dataset, {\tt Prings05},
contains eight series of tree-ring widths (mm year$^{-1}$) of maritime
pine (\textit{Pinus pinaster}), with recorded years spanning from 1810
to 2005. Ring widths in such a dataset were measured by processing
image sections of polished-core samples (5 mm diameter) with
R-package: {\tt measuRing} \citep{Lara2015}. The cores were sampled
from dominant trees of two sites, with sample plots being located on
both: northern Spain (plot code: P44005) and center-east portion of
the same country (plot code: P16106). Two trees were selected by plot,
and two core samples were extracted by tree. Consequently, the sample
design defined three levels: sample in tree on plot (plot level),
sample in tree (tree level), and sample level. 

The second data set, {\tt PTclim05}, consists of two series of monthly
mean temperatures ($^{\circ}$C month$^{-1}$) and monthly precipitation
sums (mm month$^{-1}$), with recorded months beginning at January and
ending at December, and observed years spanning from 1951 to
2005. Climatic series are labeled with the code names of correspondant
plots, with labels defining one sample level (plot).

Besides these MDFs, {\tt numeric} vector {\tt Pradii03} in examples of
the package contains inside-bark radii, measured at trees in described
plots in 2003 (mm). This radial vector was obtained from the over-bark
diameters at breast height (cm), by developing allometric estimation
of inside-bark diameters from the over-bark diameters. Names of the
vector correspond to labels in {\tt Prings05} at tree level. This
vector is used by algorithm to scale the cummulative tree growth. Some
times processed cores do not to contain initial rings from the
pits. These samples lack of information about initial states of
growth, and altere scales of the modeled tree growth.

These datasets illustrate required inputs of {\tt biodRy}, and
are used in following sections to better explain implementation of
the package.

\section{Algorithm description}
R-package {\tt biodRy} holds two function types to process the MDFs:
one-level functions, and multilevel functions. The one-level functions
process recorded series at one specific sample level: i.e., tree rings
in a sample replicate, or montly mean temperatures recorded during a
year.  However, these simple functions can only be used to process
MDFs if they are implemented by multilevel functions. Both kind of
functions are explained in following sections.

\subsection{One-level functions}\label{sec:onelev}

One-level functions are defined here as simple routines that can only
be evaluated at one specific level. Such functions are implemented by
multilevel functions (Section \ref{sec:mlev}) to model MDFs. This
section explains most important one-level functions in the package:
{\tt rtimes}, {\tt scacum}, {\tt moveYr}, {\tt amod}, and {\tt wlai}.

Function {\tt rtimes} computes relative times from 1 to number of
observed years of a series, with names of the series being recorded
years.  These relative times can sinchronized by extracting only
duplicated times and replacing unique times with {\tt NA}. Duplicated
values are controlled with default argument {\tt only.dup =
  TRUE}. This sinchronization improves the convergence of LME models
during detrending process of the series (extraction of residuals) and
avoids biases from artificial increases/decreases in growth rates
\citep{Bowman2013}.

The second function, {\tt scacum}, implements {\tt cumsum} of
R-package {\tt base} to compute radial growth. Cummulative sums only
represent such radial growth if the tree pits has been sampled. When
these pits have not been sampled, then the cummulative sums should be
scaled on a reference inside-bark radious measured at any year during
the sampled spam. Reference radious and measured year can be specified
with arguments {\tt y} and {\tt z}, respectively.

Third function {\tt amod} uses parameters of simple allometric model:
$y=a(2x)^b$, with $a,b$ being constants, and $x$ being
scaled-cummulative radial increments, to derive other dendrometric
variables. Such parameters are controled with argument {\tt mp}. For
instance, default {\tt mp = c(0.5,1)} maintains the radial increments;
{\tt mp = c(1,1)} produces diameters; and {\tt mp = c(0.25 * pi,2)}
computes basal areas. Besides, argument {\tt mp} can have more than
two parameters: $c(a_1,b_1,a_2,b_2,\cdots, a_n,b_n)$, with $n$ being
the number of times that allometric model is recursively
implemented. This recursive evaluation is useful to derive variables
which depend on other allometric covariables: i.e parameters in two
allometric models would be implemented to recursively compute
over-bark diameters and then tree biomasses. After dendrometric
variable is computed, the increments of such a variable are recomputed
by implementing {\tt setdiff} of R-package {\tt base}. Finally, SI
units of processed cummulative-radial increments can be transformed
with argument {\tt un = c(from, to)}, with available units defined in
Table (\ref{tab:gmod}).

Function {\tt moveYr} computes seasonal years from calendar
years. This function processes a {\tt numeric} vector of repeated
years with vector names being {\tt character} months in {\tt
  month.abb} or {\tt month.name} (R-package base), or a {\tt
  data.frame} object with factor-level columns of months and years,
for each of such years can begin in a month different than {\tt
  January}.

Final function {\tt wlai} computes annual aridity indexes from
Walter-Lieth diagrams (Fig). Areas between temperature and
precipitation lines when precipitation exceeds temperature are
calculated as indicators of moist seasons, and areas where temperature
exceeds precipitation are calculated as indicator of dry season. The
aridity index is defined as the quotient between the areas of dry and
wet seasons. Those precipitations over 100 mm are scaled such that 1
degree C is equal to 5 mm.

\subsection{Multilevel functions}\label{sec:mlev}

Multilevel functions are defined in this paper as those functions
which operate on MDFs.  Package {\tt biodRy} implements four kind of
multilevel functions: {\tt ringApply}, {\tt modelFrame}, {\tt
  ringLme}, and {\tt muleMan}. These functions are explained below to
get a better understanding of the package. However, most of the
procedures developed by these functions can be implemented with ony
two of them : {\tt modelFrame} and {\tt muleMan}.

Function {\tt ringApply} is a wrapper of the higher-order function
{\tt Map} (R-package base) aimed to apply individual one-level
functions at specific levels in the MDF, and preserve initial
factor-level structure in the outputs. Function arguments should be
formulated as suggested in {\tt mapply}, with no vectorized arguments
being stored in a {\tt MoreArgs} list. For instance:

\begin{verbatim}
R:> data(PTclim05)
R:> climOct <- ringApply(
                         rd = PTclim05,
                         lv = 'year',
                         fn = 'moveYr',
                         MoreArgs = list(
                                         ini.mnt = 'Oct'           
                                        ),
                        )
\end{verbatim}

where {\tt rd} is the processed MDF; {\tt lv} is either the name or
number position of the evaluated level; {\tt fn} is the implemented
one-level function; and, {\tt MoreArgs} contains arguments of the
one-level function. In this case we have computed seasonal years in
MDF {\tt PTclim05}, with years beginning at October.

Recursive function {\tt modelFrame} implements {\tt ringApply} by
default to simultaneously evaluate diverse one-level functions on the
MDF (see example in Section \ref{sec:growth}). Consequently, arguments
not to be vectorized should also be specified in a {\tt MoreArgs}
list.

Modeled MDFs from {\tt modelFrame} can be normalized (detrending of
series) by fitting linear mixed-effects models with function {\tt
  ringLme}. Such a routine implements methods from R-package {\tt
  nlme}. Two kind of model formulas are available: 'lmeForm' and
'tdForm', these characters implement functions with same names (see
manual of R-package {\tt biodRy}). Users can develop their own LME
formulas by modifying arguments in any of these methods. After LME
models are fitted, they can be extended by modeling heteroscedasticity
and autocorrelation of the residuals. Nevertheless, such residual
modeling would take long time depending on the complexity of the
modeled levels.

Two objects from {\tt modelFrame} which intersect in a recorded span
and have one sample level in common can be compared with multilevel
function {\tt muleMan}. This function develops mantel correlograms by
level by implementing methods from R-package {\tt ecodist}.

\section{Multilevel modeling}

Package {\tt biodRy} involves diverse routines to process
MDFs. Nevertheless, only two functions are necessary to model
dendroclimatical interactions: the recursive function {\tt
  modelFrame}, and the comparative function {\tt muleMan}. Former
recursive function is used to model MDFs and the comparative function
is implemented to compare the modeled MDFs.

Most of the parameters in {\tt modelFrame} have defaults, being only
required specification of an MDF in argument {\tt rd}. Nevertheless,
this function has four optional parameters: three {\tt list}
arguments: {\tt lv}, {\tt fn}, and {\tt MoreArgs}, and a {\tt
  character} method in argument {\tt form}. These {\tt list} arguments
make algorithm to recursively evaluate one-level functions in {\tt fn}
at levels in {\tt lv} with parameters of the functions defined in {\tt
  MoreArgs}. The {\tt character} method {\tt form} is used to
normalize the modeled series by detrending such series with linear
mixed-effects models. If the above mentioned arguments and are not
specified, function {\tt modelFrame} uses default arguments in
packageto to model radial increments without normalization of series.


Following sections explain use of such functions.

\subsection{Tree-growth modeling}\label{sec:growth}

Modeling of tree growth is carried out with recursive function {\tt
  modelFrame}. This routine uses allometric parameters to model
several tree-growth variables: tree diameters, basal areas, tree
biomasses, etc., and to detrend the modeled tree growth. Overall
procedure can be controlled with diverse parameters (Tables
\ref{tab:onefn}, and \ref{tab:gmod}), with number of required
arguments depending on complexity of the model. For instance, the
routine:

\begin{verbatim}
R:> data(Prings05)
R:> data(Pradii03)
R:> diamod <- modelFrame(
                         rd = Prings05,
                         lv = list(2,1,1),
                         fn = list('rtimes','scacum','amod'),
                         MoreArgs = list(
                                         z  = 2003,           
                                         mp = c(1,1),         
                                         un = c('mm','cm')    
                                        ),
                         y  = Pradii03,
                         form = 'tdForm'              
                       )
\end{verbatim}

produces object {\tt diamod}: a model of detrended diametric growth
(cm), by implementing three one-level functions: {\tt rtimes}, {\tt
  scacum}, and {\tt amod} on the MDF {\tt Prings05}. 

%% Besides the {\tt Prings05} MDF contained in argument {\tt rd}, five
%% additional parameters are required: three {\tt list} arguments ({\tt
%%   lv}, {\tt fn}, and {\tt MoreArgs}), one {\tt numeric} vector ({\tt
%%   y}), and a detrending formula ({\tt form}).

%% The above mentioned {\tt list} arguments make algorithm to recursively
%% evaluate functions in {\tt fn} at levels in {\tt lv} with parameters
%% defined in {\tt MoreArgs}. 
For the case of the example, the function
{\tt rtimes} is evaluated at the second position of factor-column
levels in {\tt Prings05} ('tree' level), and both: {\tt scacum} and
{\tt amod} are evaluated at first position of such factor columns
('sample' level). The {\tt MoreArgs} list in {\tt diamod} contains the
arguments of functions in {\tt fn} that do not change during the
within-level evaluation of the functions. If any of these arguments
changes in evaluated levels (parameters to vectorize over) then it
should be formulated outside the {\tt MoreArgs} list.

Vector {\tt y = Pradii03} contains the reference radii measured on
2003. As a different radious was measured on each tree, this argument
was formulated outside the {\tt moreArgs} list. An important feature
of the recursive function {\tt modelFrame} is that reference radii can
be vectorized even if radii were not measure at lowest level. Such is
the case of the example, where the reference radii where measured at
tree level.

Argument {\tt form} is used to specify a method to normalize the
modeled growth. Two methods are available in the package: 'tdForm' and
'lmeForm'. Former method implements a semilogarithmic model according
to linear generalization of growth from \cite{Zeide1993}, and second
method fits a simple linear model. Parameters in model methods can be
changed by specification of other argumets in each of the formulas
with similar names (see manual of R-package: {\tt biodRy}), In such a
cases, the formula argumets should be formulated outside the {\tt
  moreArgs} list.  When {\tt form = NULL} then modeled growth is not
detrended.


\subsection{Aridity modeling}\label{sec:armod}

Modeling of drought from MDF {\tt PTclim05} can also be developed with
function {\tt modelFrame}. Formulation of parameters is similar to
that explained in the modeling of tree growth (see Section
\ref{sec:growth}), but two new one-level functions are assessed at a
common temporal level:

\begin{verbatim}
  R:> drought <- modelFrame(
                            rd = PTclim05,
                            lv = list('year','year'),
                            fn = list('moveYr','wlai'),
                            form = 'lmeForm'              
                           )
\end{verbatim}

 Functions {\tt moveYr} and {\tt wlai} computes seasonal years from
 default month 'October' and calculates annual aridity indexes from
 Walter-Lieth diagrams for each seasonal year. Such functions were
 evaluated at the temporal level 'year'. As default values in
 functions were maintained, there was no necessity to specify required
 arguments in a {\tt MoreArgs} list.

\subsection{{\tt ModelFrame} outputs}
Evaluation of function {\tt modelFrame} produces either a data frame
or a list. The model is a data frame when argument {\tt form =
  NULL}. This data frame contains the processed series of growth and
preserves the temporal and factorial levels of initial MDF. When {\tt
  form = 'tdForm'} or {\tt form = 'lmeForm'} then output es an {\tt
  lmeObject}: a list containing all the parameters of an LME model.
dataset.



\section{Multilevel comparison}

{\tt biodRy} implements multivariate comparison of the modeled MDFs by
computing similarity matrices and comparing such matrices with Mantel
correlograms with function {\tt muleMan}.

\subsection{{\tt muleMan} outputs}
Evaluation of function {\tt modelFrame} produces either a data frame
or a list. The model is a data frame when argument {\tt form =
  NULL}. This data frame contains the processed series of growth and
preserves the temporal and factorial levels of initial MDF. When {\tt
  form = 'tdForm'} or {\tt form = 'lmeForm'} then output es an {\tt
  lmeObject}: a list containing all the parameters of an LME model.
dataset.
\section{Acknowledgments}

We are thankful to researchers of the Sustainable Forest Management
Research Institute, UVa-INIA for providing support in core sampling
and for giving ideas related to the development of the algorithm; in
particular Cristobal Ordo\~nez, and Ana de Luca. We also would like to
thank the anonymous reviewers for their suggestions and
comments. This research has been supported with funds from the
Colombian government (Colciencias) within the framework of the
National Program for Doctoral Studies Abroad (year 2012, number 568).

\newpage
\section{\refname}
\bibliographystyle{model2-names}
\bibliography{bPaper.bib}

\clearpage
\begin{figure}\centering
\includegraphics[scale=1.15,trim=20mm 0mm 20mm 0mm]{bfdots}
\caption{Arguments in functions of {\tt biodRy}. The ellipsis term
  means that a function can pass arguments to the previous function.}
\label{fig:fdots} 
\end{figure}

\clearpage
\begin{figure}\centering
\includegraphics[scale=0.75,trim=20mm 0mm 20mm 0mm]{diam} 
\caption{Normalized diameters at sample level.}
\label{fig:diam} 
\end{figure}

\clearpage
\begin{figure}\centering
\includegraphics[scale=0.75,trim=20mm 0mm 20mm 0mm]{arid} 
\caption{Normalized Walter-Lieth aridity indexes at plot level.}
\label{fig:fdots} 
\end{figure}

\clearpage
\begin{figure}\centering
\includegraphics[scale=0.75,trim=20mm 0mm 20mm 0mm]{mant} 
\caption{Mantel correlograms at plot level.}
\label{fig:mant} 
\end{figure}



\clearpage
\begin{table}
\scriptsize
\centering
\caption{One-level functions to be evalluated in {\tt
    modelFrame}. Five functions are available: {\tt amod}: allometric
  modeling. {\tt moveYr}: seasonal years. {\tt rtimes}: sinchronized
  relative times. {\tt sacum}: scaled-cummulative sums. {\tt wlai}:
  Walter-Lieth aridity index. See examples in R-manual of {\tt biodRy}
  to use these functions independently.}
\label{tab:onefn}
\renewcommand{\arraystretch}{1.4}
\begin{tabular}{llp{8.5cm}}
\toprule 
Function & Arguments & Definition in {\tt biodRy} manual\\ 
\midrule 

%% \verb@amod@ & {\tt cs} & {\tt numeric} vector of scaled-cummulative
%% sums such as that produced by {\tt scacum}. \\

\verb@amod@ & {\tt mp} & {\tt numeric}. vector with allometric
parameters. Different dendrometric variables can be computed; for
example, default {\tt c(0.5,1)} maintains the original radii, {\tt
  c(1,1)} produces diameters, and {\tt c(0.25 * pi,2)} computes basal
areas. This argument can have more than two parameters: {\tt
  c(a1,b1,a2,b2, ..., an,bn)}, with {\tt n} being the number of times
that allometric model will be recursively implemented. \\

 & {\tt un} & {\tt NULL}, or bidimensional {\tt character} vector with
the form {\tt c(initial, final)} to transform SI units of the
processed variable. The SI units can be expressed in micrometers {\tt
  'mmm'}, milimeters {\tt 'mm'}, centimeters {\tt 'cm'}, decimeters
{\tt 'dm'}, or meters {\tt 'm'}. If {\tt NULL} then original units are
maintained. \\

%% \verb@levexp@ & {\tt x} & numeric vector with names of the vector
%% representing the levels to be matched. \\

%%  & {\tt levels} & {\tt data.frame} with factor-level columns, or
%% {\tt character} vector of levels. \\

%% \verb@moveYr@ & {\tt cd} & {\tt data.frame} object with
%% factor-level columns of months and years, or numeric vector of
%% repeated years with vector names being the months. \\

\verb@moveYr@ & {\tt ini.mnt} & {\tt character} of initial month of
the year with character months being defined either in {\tt month.abb}
or {\tt month.name} (R-package {\tt base}), or numeric value from 1
to 12. Default 'Oct' produces years to begin in October. \\

%% \verb@rtimes@ & {\tt x} & numeric vector with names representing the
%% formation years, or multilevel data frame containing a column of
%% years.\\

 \verb@rtimes@ & {\tt only.dup} & {\tt logical}. Extract only relative
 times that are duplicated. If {\tt TRUE} then unique times are
 replaced with {\tt NA}. If all computed times are unique then this
 argument is ignored. \\

%% \verb@scacum@ & {\tt x} & numeric vector with vector names being
%% ordered on time.\\

\verb@scacum@ & {\tt y} & {\tt numeric} constant or vector to scale
the vector. Cummulative values of the numeric vector are first
computed with cumsum, and then scaled on this value. If {\tt NA} then
such cummulative sums are not scaled.\\

 & {\tt z} & {\tt NA} or numeric constant in range of the vector names
to sqcale the cummulative values. If {\tt NA} then maximun value in
such a range is used.\\

%% \verb@wlai@ & {\tt cd} & {\tt data.frame} object with annual
%% climatic records of monthly precipitation sums (mm month$^{-1}$),
%% and monthly average temperatures ($^{\circ}$C month$^{-1}$), with
%% row names being month names. \\

\verb@wlai@ & {\tt sqt} & {\tt logical}. Print the square root of the
aridity index. If {\tt TRUE} then computed aridity index is normalized
with a square root transformation. \\

 & {\tt fig} & {\tt logical}. Plot the Walter-Lieth diagram.\\

 & $\cdots$ & Further arguments to be passed to {\tt plot}.\\

\bottomrule \hline
\end{tabular}
\end{table}

\clearpage
\begin{table}
%% \renewcommand{\arraystretch}{1.9}
\scriptsize
\centering
\caption{Arguments in {\tt modelFrame} used to model diametric growth
  (cm).  The object {\tt Prings05} correspond to a multilevel data
  frame with annual tree-ring widths (mm), one temporal level: 'year',
  and three sample levels: 'sample', 'tree', and 'plot'. Bullet symbol
  (\textsuperscript{\textbullet}) indicate default arguments in
  functions, and dagger symbol (\textsuperscript{\textdagger})
  indicate arguments that should be formulated in a 'MoreArgs'
  list. Names of parent functions are written, in alphabetical order,
  among brackets [ ] (see manual of R-package: {\tt biodRy} for more
  details about such functions).}  \renewcommand{\arraystretch}{1.4}
\label{tab:gmod}
\begin{tabular}{lp{9.5cm}}
\toprule 
Arguments &Description in R manual. [ parent function(s) ].\\  
\midrule

{\tt rd = Prings05} & \indent {\tt data.frame} object with tree-ring
series (TRS) and factor levels (FL). [ {\tt arguSelect}, {\tt levexp},
  {\tt lmeForm}, {\tt ringApply}, {\tt ringLme}, {\tt shiftFrame},
  {\tt splitFrame}, {\tt tdForm} ].\\

{\tt lv = list(2,1,1)} \textsuperscript{\textbullet}&\indent List of
{\tt numeric} positions in FL, or {\tt character} names of such
columns, for recursive implementation of functions in {\tt fn}. [ {\tt
    ringApply}, {\tt splitFrame} ].\\

{\tt fn} \textsuperscript{\textbullet}&\indent Default {\tt fn =
  list('rtimes','scacum','amod')}. List of {\tt character} nameqs of
functions to be recursivelly implemented. [ {\tt ringApply} ].\\

{\tt only.dup = TRUE} \textsuperscript{\textbullet}&\indent {\tt
  logical}. Sinchronize the computed relative times by replacing
unique values with {\tt NA}. [ {\tt rtimes} ].\\

{\tt y = Pradii03}&\indent {\tt NA} or {\tt numeric} vector to scale
the TRS. If {\tt NA} then cummulative sums are computed but not
scaled. [ {\tt scacum} ].\\

{\tt z = 2003} \textsuperscript{\textdagger}&\indent {\tt NA} or {\tt
  numeric} constant in range of the vector names. If {\tt NA} then
maximun value in such a range is used. [ {\tt scacum} ].\\

{\tt form = 'tdForm'}&\indent {\tt character} or {\tt {\tt
    NULL}}. Name of a linear-mixed-effects formula (LME formula) used
to detrend the modeled tree growth. [ {\tt ringLme} ].\\

%% {\tt prim.cov = FALSE}&\indent {\tt logical}. Implement a primary
%% covariate form: '~ cov'. If {\tt FALSE} then a complete formula:
%% 'resp ~ cov | group' is implemented. [ {\tt tdForm} ]\\

{\tt on.time = TRUE} \textsuperscript{\textbullet}&\indent {\tt
  logical}. If {\tt TRUE} then detrending is developed on the relative
time. If {\tt FALSE} then the observed years are used. [ {\tt tdForm}
].\\

{\tt log.t = FALSE} \textsuperscript{\textbullet}&\indent {\tt
  logical}. If {\tt FALSE} then f(t) = t. If {\tt TRUE} then f(t) =
ln(t). [ {\tt ringLme}, {\tt tdForm} ].\\

%% {\tt lev.rm = NULL} \textsuperscript{\textbullet}&\indent {\tt {\tt
%% NULL}} or {\tt character}. Name of factor-level column to be
%% removed from the formula. [ {\tt tdForm} ].\\
 
{\tt mp = c(1,1)} \textsuperscript{\textdagger}&\indent {\tt
  numeric}. Allometric parameters to compute tree diameters from
TRS. [ {\tt amod} ].\\

{\tt un = c('mm','cm')} \textsuperscript{\textdagger}&\indent {\tt
  {\tt NULL}} or bidimensional {\tt character} to transform SI units
of the processed variable. The SI units can be expressed in
micrometers {\tt 'mmm'}, milimeters {\tt 'mm'}, centimeters {\tt
  'cm'}, decimeters {\tt 'dm'}, or meters {\tt 'm'}. If {\tt NULL}
then original units are maintained.[ {\tt amod} ].\\

{\tt var.mod = FALSE} \textsuperscript{\textbullet}&\indent {\tt
  logical}. If {\tt TRUE} then the fitted model is extended by
modeling residual heteroscedaticity with 'varConstPower' class in {\tt
  nlme}. [ {\tt ringLme} ].\\

{\tt arma.mod = FALSE} \textsuperscript{\textbullet}&\indent {\tt
  logical}. If {\tt TRUE} then the fitted model is extended by
modeling residual autocorrelation with class 'corARMA' in {\tt nlme},
with {\tt p = 1} and {\tt q = 1}, which models residual
autocorrelation for lags $\geq2$. [ {\tt ringLme} ].\\

{\tt res.data = TRUE} \textsuperscript{\textbullet}&\indent {\tt
  logical}. If {\tt TRUE} then a data frame of name 'resid' is added
to output list. [ {\tt ringLme} ].\\

\bottomrule %% \hline
\end{tabular},
\end{table}



\end{document}
